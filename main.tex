\documentclass{article}
\usepackage[UTF8]{ctexcap}
\usepackage{hyperref}
\usepackage{amsmath}
\usepackage{amssymb}
\usepackage{geometry}
\geometry{a4paper,left=3cm,right=3cm,top=4cm,bottom=4cm}

\hypersetup{
    colorlinks=true,
    linkcolor=blue,
    filecolor=magenta,
    urlcolor=cyan,
}

\urlstyle{same}

\title{周二计量作业}
\author{outsider}
\date{March 2019}

\begin{document}

\maketitle

\begin{enumerate}
    \item \textbf{证明条件期望的迭对定律。}Green P1045

    来源:唐晓彬课件,数学知识回顾,第五页PDF

    \textbf{Proof}

    离散情况

    $$
      \begin{aligned}
        & E_X[E(Y|X)] = \sum_{i}P(X=x_i)E(Y|x_i) \\
        & = \sum_iP(X=x_i)[\sum_jP(Y=y_i|x_i)y_j] \\
        & = \sum_iP(X=x_i)[\sum_j\frac{P(X=x_i, Y=y_i)}{P(X=x_i)} y_j] \\
        & = \sum_i[\sum_jP(X=x_i, Y=y_j) y_j] \\
        & = \sum_j[\sum_iP(X=x_i, Y=y_j) y_j] \ \ \ \ \ \ \ \ \ \ \text{交换加总次序}\\
        & = \sum_j[y_j \sum_iP(X=x_i, Y=y_j)] \\
        & = \sum_jy_jP(Y=y_j) \\
        & = E(Y)
      \end{aligned}
    $$

    连续情况

    $$
      \begin{aligned}
        \because & E_X(g) = \int_{x} gf(x)dx\ \ \ \ and \ \ \ \ E(y|x) = \int_yyf(y|x)dy \\
        \therefore & E_X[E(Y|X=x)] = \int_x [E(Y|X=x)]f(x) dx \\
        & = \int_x\left( \int_yyf(y|x)dy \right) f(x)dx \\
        & = \int_x \int_y yf(y|x)dy f(x)dx \\
        & = \int_x \int_y yf(y|x)f(x)dxdy \\
        & = \int_x \int_y yf(x, y)dxdy \\
        & = \int_y yf(y)dy = E(y)
      \end{aligned}
    $$

    \item \textbf{证明课件Th3.3 Frisch-Waugh(Ch1-P10)。} In the least squares regression of vector $y$ on two set of variables. $X_1$ and $X_2$, the subvector $b_2$ is the set of coefficients obtained when the residuals from a regression of $y$ on $X_1$ alone are regressed on the set of residuals obtained when each column of $X_2$ is regressed on $X_1$.

    来源:Econometrics Analysis

    \textbf{Proof}

    $$
    y=X\beta+\epsilon=X_1\beta_1+X_2\beta_2+\epsilon
    $$
    进而
    $$
    \left[
    \begin{array}{cc}
      X_1'X_1 & X_1'X_2 \\
      X_2'X_1 & X_2'X_2 \\
    \end{array}
    \right]
    \left[
    \begin{array}{c}
      b_1 \\
      b_2
    \end{array}
    \right]
    =
    \left[
    \begin{array}{c}
      X_1'y \\
      X_2'y
    \end{array}
    \right]
    $$

    求解出$b_1$

    $$
    b_1=(X_1'X_1)^{-1}X_1'y-(X_1'X_1)^{-1}X_1'X_2b_2==(X_1'X_1)^{-1}X_1'(y-X_2b_2)
    $$

    并且有

    $$
    X_2'X_1b_1+X_2'X_2b_2=X_2'y
    $$

    将$b_1$带入,有

    $$
    X_2'X_1(X_1'X_1)^{-1}X_1'y-X_2'X_1(X_1'X_1)^{-1}X_1'X_2b_2+X_2'X_2b_2=X_2'y
    $$

    整理后可得

    $$
    \begin{aligned}
    b_2 =& [X_2'(I-X_1(X_1'X_1)^{-1}X_1')X_2]^{-1}[X_2'(I-X_1(X_1'X_1)^{-1}X_1')y] \\
    =& (X_2'M_1X_2)^{-1}(X_2'M_1y)
    \end{aligned}
    $$

    令

    $$
    \left\{
    \begin{aligned}
    X_2^{*'} &= M_1X_2 \\
    y^* &= M_1y \\
    \end{aligned}
    \right.
    $$
    
    \noindent\rule{\textwidth}{0.4pt}

        来源:周一计量课上的证明

        \textbf{Proof}

        先证系数相等

        $$
          \begin{aligned}
            X_1'Y & = X_1'(X_1\widehat{\beta_1}+X_2\widehat{\beta_2}+e) \\
            & = X_1'X_1\widehat{\beta_1}+X_1'X_2\widehat{\beta_2}+X_1'e \\
            & = X_1'X_1\widehat{\beta_1}+X_1'X_2\widehat{\beta_2} \ \ \ \ \ \ \ (1)
          \end{aligned}
        $$

        同理可得

        $$
          \begin{aligned}
            X_2'Y = X_2'X_1\widehat{\beta_1} + X_2'X_2\widehat{\beta_2} \ \ \ \ \ (2)
          \end{aligned}
        $$

        令(1)式左乘$X_1(X_1'X_1)^{-1}$

        $$
          \begin{aligned}
            & X_1(X_1'X_1)^{-1}X_1'Y \\
            = & X_1(X_1'X_1)^{-1}X_1'X_1\widehat{\beta_1} + X_1(X_1'X_1)^{-1}X_1'X_2\widehat{\beta_2} \\
            = & X_1\widehat{\beta_1}+P_{X_1}X_2\widehat{\beta_2} = P_{X_1}Y
          \end{aligned}
        $$

        考虑到

        $$
          \begin{aligned}
            X_1\widehat{\beta_1} = & P_{X_1}Y - P_{X_1}X_2\widehat{\beta} \\
            = & P_{X_1}(Y-X_2\widehat{\beta}) \ \ \ \ \ \ \ \ \ \ \ (3)
          \end{aligned}
        $$

        将(3)式代入(2)式,得到

        $$
          \begin{aligned}
            X_2'Y &= X_2'P_{X_1}Y - X_2'P_{X_1}X_2\widehat{\beta_2}+X_2'X_2\widehat{\beta_2} \\
            X_2'(I-P_{X_1})Y &= X_2'(I-P_{X_1})X_2\widehat{\beta_2} \\
            X_2'M_{X_1}Y &= X_2'M_{X_1}X_2\widehat{\beta_2} \\
            \widehat{\beta_2} &= (X_2'M_{X_1}X_1)^{-1}(X_2'M_{X_1}Y)
          \end{aligned}
        $$

        再证残差项等

        $$
          e = T - _1\widehat{\beta_1}-X_2\widehat{\beta_2}
        $$

        两边同乘$M_{X_1} = I-P_{X_1}$,利用$P_{X_1}e=0$

        $$
          \begin{aligned}
            e &= (I-P_{X_1})Y-(I-P_{X_1})X_1\widehat{\beta_1} - (I-P_{X_1})X_2\widehat{\beta_2} \\
            &= Y^* - X_2^*\widehat{\beta_2}=e^*
          \end{aligned}
        $$

        如何得到$b_1$

        $$
          b_1 = (X_1'X_1)X_1'Y-(X_1'X_1)^{-1}X_1'X_2\widehat{\beta_2}
        $$

    \item \textbf{证明BLUE中的有效性。}令$b^{*}=\beta+(X'X)^{-1}X'\epsilon+C\epsilon$,其中$CX=0$,证明$Var(b^*|X)$

    来源:OUTSIDER

    \textbf{Proof}

    $$
      \begin{aligned}
        &Cov((X'X)^{-1}X'\epsilon,\ C\epsilon) \\
        = &E[((X'X)^{-1}X'\epsilon)(C\epsilon)] - E[(X'X)^{-1}X'\epsilon]E[C\epsilon] \\
        = &E[((X'X)^{-1}X'\epsilon)(C\epsilon)] \ \ \ \ \ \ \ \ C\epsilon\text{是一个数}\\
        = &E[(X'X)^{-1}X'\epsilon\epsilon'C'] \\
        = &\sigma^2E[(X'X)^{-1}X'C'] \\
        = &\sigma^2E[(X'X)^{-1}(CX)'] = 0
      \end{aligned}
    $$

    \noindent\rule{\textwidth}{0.4pt}

    补BLUE有效性的证明

    来源:Econometrics Analysis P60

    \textbf{Proof}

    定义$b_0=Cy$是另一个对$\beta$的线性无偏估计量,则有

    $$
    E[Cy|X] = E[(CX\beta+C\epsilon)|X] = \beta
    $$

    可以证明$CX=I$成立,$b_0$的条件方差为

    $$
    Var[b_0|X] = \sigma^2CC'
    $$

    定义$D=C-(X'X)^{-1}X'$,则有

    $$
    Var[b_0|x] = \sigma^2[(D+(X'X)^{-1}X')(D+(X'X)^{-1}X')']
    $$

    由于$CX=I=DX+(X'X)^{-1}(X'X)$,因此,必须有$DX=0$,所以

    $$
    Var[b_0|X] = \sigma^2(X'X)^{-1}+\sigma^2DD'=Var[b|X]+\sigma^2DD'
    $$

    由于$DD'=q'DD'q=z'z\ge 0$,所以有$Var[b_0|X] \ge Var[b|X]$

    所以$b$是

    \item 回归方程$y_i=\beta_0+\beta_1X_{i1}+\beta_2X_{i2}+\cdots+\beta_KX_{iK}+\epsilon_i$存在异方差的问题,假定残差的方差为$\sigma_i^2=\sigma^2X_{ik}^2$。将回归方程重写为
    $$
      \frac{y_i}{X_{ik}}=\frac{\beta_0}{X_{ik}}+\beta_1\frac{X_{i1}}{X_{ik}}+\cdots+\beta_K\frac{X_{iK}}{X_{ik}}+\frac{\epsilon_i}{X_{ik}}
    $$
    按照加权最小二乘估计(WLS)进行参数估计。证明WLS是GLS的一种特殊情况。

    来源:OUTSIDER

    \textbf{Proof}

    定义$\Omega$

    $$\Large\Omega=
    \Large\begin{array}{cccc}
      \frac{1}{x_{ik}} & 0 & \cdots & 0 \\
      0 & \frac{1}{x_{ik}} & \cdots & 0 \\
      0 &   & \ddots & \vdots \\
      0 & 0 & \cdots & \frac{1}{x_{ik}} \\
    \end{array}
    $$

    将$\Omega^{-1/2}$乘到回归模型的两端,得到

    $$
    \Omega^{-1/2}y=\Omega^{-1/2}X\beta+\Omega^{-1/2}\epsilon
    $$

    令

    $$
    \left\{
      \begin{aligned}
        y^* &= \Omega^{-1/2}y \\
        X^* &= \Omega^{-1/2}X \\
        \epsilon^* &= \Omega^{-1/2}\epsilon \\
      \end{aligned}
    \right.
    $$

    则有

    \begin{itemize}
      \item $E(\epsilon^*|X)=0$
      \item $Var(\epsilon^*|X) = \Omega^{-1/2}Var(\epsilon)\Omega^{-1/2}=\sigma^2\Omega^{-1/2}\Omega\Omega^{-1/2}=\sigma^2I$
    \end{itemize}

    所以

    $$
    \begin{aligned}
      b &= (X^{*'}X^*)^{-1}X^{*'}y^* \\
      &= (X'\Omega^{-1/2}\Omega^{-1/2}X)^{-1}(X'\Omega^{-1/2})\Omega^{-1/2}y \\
      &= (X'\Omega^{-1} X)^{-1}(X'\Omega^{-1}y)
    \end{aligned}
    $$

    \item 回归模型服从以下方程
    $$
    \left\{
      \begin{aligned}
        y_t&=\alpha+\beta x_t+\epsilon_t \\
        \epsilon_t&=\rho \epsilon_{t-1}+\epsilon_t^{*} \\
        \epsilon_t^{*} &\sim \text{白噪声}\\
      \end{aligned}
    \right.
    $$
    因此将上述变量作广义差分处理,如下
    $$
    \left\{
      \begin{aligned}
        y_t^* &= y_t-\rho y_{t-1} \\
        x_t^* &= x_t-\rho x_{t-1}  \\
        \epsilon_t^* &= \epsilon_t-\rho \epsilon_{t-1} \\
      \end{aligned}
    \right.
    $$
    并对如下模型进行参数估计
    $$
      y_t^* = \widetilde{\alpha}+\widetilde{\beta}x_t^*+\epsilon_t^*
    $$
    证明这仍是一种特殊的GLS

    参考 于俊年 计量经济学 第138页和第156页的内容
\end{enumerate}

\section*{Miscellaneous}

这一部分主要填充一些各种老师上课时透露的题目,包括以往年的题目。

\begin{enumerate}
    \item \textbf{线性变形对回归的影响。}证明:将数据的第一行和最后一行交换位置,结果不变。
\end{enumerate}
\end{document}
